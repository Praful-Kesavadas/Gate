%iffalse
\let\negmedspace\undefined
\let\negthickspace\undefined
\documentclass[journal,12pt,twocolumn]{IEEEtran}
\usepackage{cite}
\usepackage{amsmath,amssymb,amsfonts}
\usepackage{graphicx}
\usepackage{textcomp}
\usepackage{xcolor}
\usepackage{txfonts}
\usepackage{listings}
\usepackage{enumitem}
\usepackage{mathtools}
\usepackage{gensymb}
\usepackage{comment}
\usepackage[breaklinks=true]{hyperref}
\usepackage{tkz-euclide} 
\usepackage{listings}
\usepackage{gvv}                                        
\def\inputGnumericTable{}                                 
\usepackage[latin1]{inputenc}                                
\usepackage{color}                                            
\usepackage{array}                                            
\usepackage{longtable}                                       
\usepackage{calc}                                             
\usepackage{multirow}                                         
\usepackage{hhline}                                           
\usepackage{ifthen}                                           
\usepackage{lscape}
\usepackage[export]{adjustbox}

\newtheorem{theorem}{Theorem}[section]
\newtheorem{problem}{Problem}
\newtheorem{proposition}{Proposition}[section]
\newtheorem{lemma}{Lemma}[section]
\newtheorem{corollary}[theorem]{Corollary}
\newtheorem{example}{Example}[section]
\newtheorem{definition}[problem]{Definition}
\newcommand{\BEQA}{\begin{eqnarray}}
\newcommand{\EEQA}{\end{eqnarray}}
\newcommand{\define}{\stackrel{\triangle}{=}}
\newtheorem{rem}{Remark}

\begin{document}
\parindent 0px
\bibliographystyle{IEEEtran}

\vspace{3cm}

\title{}
\author{EE23BTECH11217 - Prajwal M$^{*}$
}
\maketitle
\newpage
\bigskip

% \renewcommand{\thefigure}{\theenumi}
% \renewcommand{\thetable}{\theenumi}


\section*{Exercise 9.1}

\noindent \textbf{12} \hspace{2pt}Write the five terms at n = 1, 2, 3, 4, 5 of the sequence and obtain the Z-transform of the series
\begin{align}
    x \brak{n} &=  -1, & n = 0 \\
    &=   \frac{x \brak{n-1}}{n}, & n > 0\\
    &=   0, & n < 0 
\end{align}

\noindent Solution:

\noindent
\begin{align}
	x \brak{1} & = \frac{x \brak{0}}{1} = -1 \\
x \brak{2} & = \frac{x \brak{1}}{2} = -\frac{1}{2} \\
	x \brak{3} & = \frac{x \brak{2}}{3} = -\frac{1}{\brak{2} \brak{3}} = -\frac{1}{6}\\
	x \brak{4} & = \frac{x \brak{3}}{4} = -\frac{1}{\brak{2}   \brak{3} \brak{4}} = -\frac{1}{24}\\
	x \brak{5} & = \frac{x \brak{4}}{5} = -\frac{1}{\brak{2} \brak{3} \brak{4} \brak{5}} = -\frac{1}{120} \\
% \end{align} \\
% % So the first five terms of the series are:
% % $$-1 , -\frac{1}{2}, -\frac{1}{6}, -\frac{1}{24},  -\frac{1}{120}$$ \\
% % The corresponding series:
% % \begin{align}
% % :wq	\sum_{n=-\infty}^{\infty} x \brak{n} & = \ldots + 0 + x\brak{0} + x \brak{1} + x \brak{2} + \ldots \\
% % 	& = \ldots + 0 + \brak{-1} +  \brak{-1} +   \brak{-\frac{1}{2} } + \ldots 
% % \end{align}
% % The nth term of the series is,
% \begin{align}
    x \brak{n} & = \frac{-1}{n!}  \brak{u \brak{n}} \label{x(n)}
\end{align}

 \begin{figure}[h]
   \centering
   \includegraphics[width=1\columnwidth]{figs/plot.png}
   \caption{Plot of x(n) vs n}
   \label{fig: 9.1.12.1}
 \end{figure}

\begin{align}
	x \brak{n} \system{Z} X \brak{z} 
\end{align}

\begin{align}
    X \brak{z} & = \sum_{n=-\infty}^{\infty} x \brak{n}   z^{-n} \\
    \notag \text{using \eqref{x(n)}, } \\
    & = \sum_{n=-\infty}^{\infty} \frac{-1}{n!}  u \brak{n}   z^{-n} \\
    & = \sum_{n=0}^{\infty} \frac{-1}{n!}   z^{-n} \\
    & = - e^{z^{-1}} &  \cbrak{z\in\mathbb{C} : z \neq 0} 
\end{align}

% For the series to converge, the ratio test must be satisfied for n $\geq$ 0
% \begin{align}
%  \lim_{{n \to \infty}}  \left| \frac{x \brak{n+1} z^{- \brak{n+1}}}{x \brak{n} z^{-n}}  \right| & <  1 \\
% \lim_{{n \to \infty}}  \left| \frac{-z^{-n-1} n!}{-z^{-n}  \brak{n+1}!} \right| & < 1\\
% \lim_{{n \to \infty}}  \left| \frac{z^{-1}}{n+1}  \right| & < 1
% \end{align}

% The condition is satisfied for $ z \neq 0 $ \\

% Hence, ROC of Z transform is 

% \begin{align}
% 	z\in\mathbb{C} : z \neq 0.
% \end{align}

\input{tables/table}

\end{document}
